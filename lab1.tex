\documentclass[a4paper,12pt,openany]{book}
\usepackage[T1]{fontenc}
%\usepackage{amsmath}
\usepackage[latin1]{inputenc}
\usepackage{amsxtra}
\usepackage{titlesec}
\usepackage{booktabs}
\usepackage{graphicx}
\usepackage[table,xcdraw]{xcolor}
\usepackage{romannum}
\usepackage{fancyhdr}
\usepackage{etoolbox}
\usepackage{float}
\usepackage{graphicx}
\usepackage{indentfirst}
\usepackage{geometry}
\geometry{
 a4paper,
 total={170mm,257mm},
 left=20mm,
 top=20mm,
 }
\pagestyle{fancy}
\fancyhf{} 
\fancyhead[RE,RO]{\thepage}
\fancyfoot[RE,RO]{P20IC013}
\fancyfoot[LE,LO]{Darsh Gajjar}


\titleformat
{\chapter} % command
[display] % shape
{\bfseries\Large\itshape} % format
{Process Control LAB Manual } % label
{0.5ex} % sep
{
    \rule{\textwidth}{1pt}
    \vspace{-0.5ex}
    \centering
} % before-code
[
\vspace{-0.5ex}%
\rule{\textwidth}{0.3pt}
] % after-code
\titlespacing*{\chapter}{5pt}{-50pt}{10pt}

%\titleformat{\section}[wrap]
%{\normalfont\bfseries}
%{\thesection.}{0.5em}{}

%\titlespacing{\section}{12pc}{1.5ex plus .1ex minus .2ex}{1pc}

\titleformat{\section}{\normalfont\fontsize{12}{15}\bfseries}{\thesection}{1.5em}{}

\setlength{\parindent}{3em}
\setlength{\parskip}{0.5em}
\begin{document}
\chapter{Experiment : 01}

\section{Title : Introduction to MATLAB/Simulink: Plot the step response of first order and second
order systems.}


\section{Apparatus :}
MATLAB/Simulink Software \par

\section{Theory :}
let us consider first order system of which has transfer function is given below
\begin{equation}
  G_1(s) = \frac{1}{1 + s}
\end{equation}
\par and second order system which transfer function is
\begin{equation}
 G_2(s) = \frac{10}{s(s+15)}
\end{equation}
\par
Now evaluate the closed loop transfer function of above systems in which step
response is given as input and negetive unity feedback is used. general closed
loop transfer function is defined as below
\[  \frac{C(s)}{R(s)} = \frac{G(s)}{1 + G(s)H(s)} \]
\par

where $G(s)$ is  feed forward gain and $H(s)$ is feedback gain, where $C(s)$
referred as controlled ouput variable and $R(s)$ as reference set point or input
variable \par

as per above equation, we get $Y_1(s)$ and $Y_2(s)$ as first order and
second order closed loop transfer function respectively \par
\begin{equation}
  Y_1(s) = \frac{1}{s+2}
\end{equation}
\begin{equation}
  Y_2(s) = \frac{10}{s^2+15s+10}
  \pagebreak
\end{equation}
\section{Observation :}
\begin{center}
  \includegraphics[width = 165mm, scale = 0.80]{lab1fig1.png}
  Figure 1.1 : Simulink model of system under consideration
\end{center}
\begin{figure}[h!]
  \begin{center}
  \includegraphics[width = 165mm, scale = 0.80]{graph1.png}
  Figure 1.2 : Step response of first order system
  \end{center}
\end{figure}
\begin{center}
  \includegraphics[width = 165mm, scale = 0.85]{graph2.png}
  Figure 1.3 : Step response of second order system
\end{center}
\section{Conclusion :}
\begin{itemize}
\item by giving step response or bounded input, we get an bounded output in both
  the systems, so they are behave as stable.
  \item first order system will settled down to final value at approx $~$2.2
    seconds when step input is given
    \item in second order system final value achieve at approx 7 seconds due to
      its Overdamped response.
\end{itemize}
\chapter{Experiment : 02}
\section{Title : Study the effect of varying system gain for a second order overdamped system, and
verifying the results using Root locus.}

\section{Aim :}
\begin{enumerate}
  \item {\bf Analyse the step responses of Second order system at different  values of $K_P$.}
\item {\bf Analyse the step response of Second order system with transportation delay with and
    without disturbances.}
 \end{enumerate} 
\section{Apparatus :}
MATLAB/Simulink Software \par
\section{Theory :}
In a first part we have $G_1(s)$ of second order system and varying the
proportional gain($K_P$) to 50, 100, 1500 values and evaluating its step
responses.
\begin{equation}
  G_1(s) = \frac{8}{s^2+50s}
\end{equation}
\par
In second part we have second order system with dead time or transportation
delay $G_2(s)$ and disturbance with delay as $G_3(s)$ defined below
\begin{eqnarray}
  G_2(s) = \frac{2e^{-5s}}{50s^2+15s+1} \\
  \\
  G_3(s) = \frac{0.3e^{-5s}}{15s+1}
\end{eqnarray}
\pagebreak
\section{Observation :}
\begin{figure}[ht!]
  \centering
\includegraphics[width = 165mm, scale = 0.85]{lab2part11.png}
   \caption{Simulink model of above system}
\includegraphics[width = 165mm, scale = 0.85]{lab2part12.png}
   \caption{Figure 2.2 : Step responses of second order system with different
     value of $K_p$}
 \end{figure}
\begin{minipage}{0.60\linewidth}
  \includegraphics[width = \linewidth]{rlocus.jpg}
    \end{minipage} \hfil
     \begin{minipage}{0.45\linewidth}
       \begin{verbatim}
#Code for plot Root locus
# in MATLAB  programme
num = [1];
den = [1 50 8];
G = tf(num,den);
rlocus(G)
\end{verbatim}
\end{minipage}
 \begin{figure}[ht!]
 \includegraphics[width = 165mm, scale = 0.85]{lab2part21.png}
 \caption{Simulink Model of second order system with time delay}
\includegraphics[width = 165mm, height = 63mm, scale = 0.95]{lab2part22.png}
\caption{Step responses of Second order system with dead time}
\end{figure}
\section{Conclusion :}
\begin{itemize}
\item In first part we have observed that on increasing the value of $K_P$ output response is reaching
at final value very fast. It is verified from Root locus that at $K_P$ = 50 we get critical damped
response while on  $K_P$ = 150, 1500 we get underdamped response i.e. having overshoot.
   \item In second part due to disturbance, output response is getting deviated from its
reference value and reaching at peak which is greater than input.
 \end{itemize}

\chapter{Experiment : 03}
\section{Title : Dynamic response of First order system with time delay}
\section{Aim :}
\begin{enumerate}
\item Plot the responses of firsr order system with given condition:
  \begin{enumerate}
  \item Exact response
  \item {\bf \Romannum{1} order Pad\'e} Approximation
  \item {\bf \Romannum{2} order Pad\'e} Approximation
  \end{enumerate}
\item Proportional only control of First order system with dead time
\item Intergral only control of Second order system with dead time
\end {enumerate}

\section{Apparatus :}
MATLAB/Simulink Software \par
\section{Theory :}
In a first part we have plant transfer function with transportation delay as $G_1(s)$
\begin{equation}
  G_1(s) = \frac{Ke^{-sT_d}}{sT_p + 1}
\end{equation}
\par
This is exact or actual transfer function, which we have to approximate using
Taylor series approximation by expanding transportation delay term into ints
factors into poles and zeros.Now First order Pad\'e approximation for a time delay
consist of half magnitude of right hand side s plane zero and half magnitude of
left hand side s plane pole, or ratio of two polynomials in 's' with with
coefficient term calculating by taylor series expansion of $e^{-s\theta}$.

\noindent First Order Pad\'e Approximation is,
\begin{center}
\scalebox{2.0}{$
  e^{-s\theta} = (\frac{1 - \frac{s\theta}{2}}{1 + \frac{s\theta}{2}})$}
\end{center}
second Order Pad\'e Aprroximation is,
\begin{center}
  \scalebox{2.0}{$
  e^{-s \theta} = \frac{1 - \frac{s \theta}{2} + \frac{s^2 (\theta)^2}{12} }{1 + \frac{s \theta}{2} + \frac{s^2 (\theta)^2}{12}}$}
\end{center}
\noindent For a first case we have $T_d$ = 5 second, $T_p$ = 10 second and K = 1, so we
have to evaluate the responses, so that
\begin{equation}
 G(s) = \frac{ e^{-5s}}{10s + 1}
\end{equation}
\Romannum{1} Order Pad\'e Approximation
\begin{equation}
  G_1(s) = \frac{1-2.5s}{25s^2+12.5s+1}
\end{equation} \\
\Romannum{2} Order Pad\'e Approximation
\begin{equation}
  G_2(s) = \frac{2.083s^2-2.5s+1}{20.83s^3+27.083s^2+12.5s+1}
\end{equation}
In second part firstly we analyse the response of G(s) at different values of
K=1.05, 1.15, 2, 4.5, 8.5 and selecting value of $T_d$ = 5 and $T_p$=1. \par
Now in second part we change value of $k_i$= 0.2,0.5,0.8,1 respectively and add
disturbance in system with time delay of 25 seconds.
\section{Observation :}
\begin{figure}[H]
  \centering
  \includegraphics[width = 165mm, scale = 0.85]{lab03part1.png}
  \caption{Simulink model of system of part 1}
\end{figure}
\pagebreak
\begin{figure}[H]
  \centering
  \includegraphics[width = 165mm, scale = 0.85]{lab03part12.png}
  \caption{step response of All system in combined axis}
   \includegraphics[width = 165mm, height = 80mm, scale = 0.85]{lab03part2a.png}
   \caption{simulink model of part 2}
   \includegraphics[width = 165mm, scale = 0.85]{lab03part2a1.png}
   \caption{step response with $K$ = 1.05}
 \end{figure}
 \begin{figure}[H]
   \includegraphics[width = 165mm, scale = 0.85]{lab03part2a2.png}
   \caption{step response with $K$ = 1.15}
   \includegraphics[width = 165mm, scale = 0.85]{lab03part2a3.png}
   \caption{step response with $K$ = 2}
   \includegraphics[width = 165mm, scale = 0.85]{lab03part2a4.png}
   \caption{step response with $K$ = 4.5}
 \end{figure}
 \begin{figure}[H]
  \includegraphics[width = 165mm, scale = 0.85]{lab03part2a5.png}
  \caption{step response with $K$ = 8.5}\\
 \includegraphics[width = 165mm, height = 110mm, scale = 0.85]{lab03part21.png}
 \caption{simulink model of system of part 3}
\end{figure}
\begin{figure}[H]
\includegraphics[width = 165mm, height=55mm, scale = 0.85]{lab03part22_ki_0_2.png}
 \caption{step response with $K_i= 0.2$}
 \includegraphics[width = 165mm, height =55mm, scale = 0.85]{lab03part23_k_i_0_5.png}
 \caption{step response with $K_i= 0.5$}
 \includegraphics[width = 165mm, height=55mm, scale = 0.85]{lab03part24_k_i_0_8.png}
 \caption{step response with $K_i= 0.8$}
 \includegraphics[width = 165mm, height=55mm, scale = 0.85]{lab03part25_k_i_1.png}
 \caption{step response with $K_i= 1$}
\end{figure}
\section {Conclusion}
\begin{itemize}
\item In First Part of system we can verify that system incorporate inverse
    response when pad\'e approximation is used, in which initial slope is
    negetive and response goes to reverse direction to the reference value.
\item  Also if we increase order of pad\'e approximation then inverse
     response increase or become double.
\item In second part as value of K (gain) is increased then its step response has
       increasing time delay and oscillation respectively.
\item Now in Third part, for analysis of integral control only we had
         inserted additional disturbance of step input after 25 sec and check
         that system will become oscillatory in first some cycle of operation as
         value of $K_i$ is increasing.
\end{itemize}

\chapter{Experiment : 04}
\section{Title : Approximation of Higher order system into first and second order systems with time
delay using Taylor series approximation and Skogestad's Half rule.}
\section{Apparatus :}
MATLAB Software
\section{Theory :}
For Approximation of system, we need to linearized higher order nonlinear model
into its equivalent first order or second order model so we can deal easily with
its dynamic and steady and state characteristics. So that we are using methods
like Taylor series and SKOGESTAD's Half rule. \par 
\noident Tailor series approximation is expressed up to second term,so that
\begin{eqnarray}
  e^{-\theta s} \approx 1 - \theta s\\
   e^{-\theta s} = \frac{1}{e^{\theta s}} \approx \frac{1}{ 1 + \theta s}
\end{eqnarray}
Now from SKOGESTAD's methods, there are some rules to transform higher order
system into first order or second order equivalent system is given below
\begin{itemize}
\item if model has multiple time constants than manipulation takes place with
    largest neglected Time constant
\item in denominator one half of neglected T is added to the existing time
      delay and other half of neglected T is added to the time constant that is
      retained
\item Third largest time constants that are smaller than largest neglected
        T will be considered \& added as time delay term
\end{itemize}
\scalebox{1.25}{\bf Part : A}
\begin{equation}
  G(s) = \frac{K(-0.1s + 1)}{(5s+1)(3s+1)(0.5s+1)}
  \end{equation}
  \noindent Derive an approximation First order plus time delay model(FOPTD) using:
  \begin{itemize}
  \item Taylor series approximation
  \item SKOGESTAD's Half rule
  \end{itemize}
  we get our transform or linearized plant transfer function as
  \begin{equation}
  G_{T}= \frac{Ke^{-3.6s}}{5s + 1},  G_H = \frac{Ke^{-2.1s}}{6.5s + 1}  
  \end{equation}
  where $G_T$ = approximated transfer function using taylor series\\
  $G_H$ =  approximated transfer function using SKOGESTAD's half rule\\
  \par
 \noindent \scalebox{1.25}{\bf Part : B}
 \begin{equation}
   H(s) = \frac{K(1 - s)e^{-s}}{(12s+1)(3s+1)(0.2s+1)(0.05s+1)}
 \end{equation}
   \noindent Derive models using Skogestad's rule:
   \begin{itemize}
   \item First order plus time delay
   \item Second order plus time delay take K = 1
 \end{itemize}
   \par
   we get transformed approximated plant transfer function as
\begin{equation}
   H_{T1}(s)= \frac{e^{-3.75s}}{13.5s + 1},  H_{H2}(s) = \frac{Ke^{-2.15s}}{(12s + 1)(3.1s + 1)}
 \end{equation}
 \section{Observation : }
\begin{center}
  \includegraphics[width=165mm, scale=0.95]{lab04part10.png}
  \caption{Simulink model of systems given is Part A}
\end{center}
  \begin{figure}[H]
    \centering
    \includegraphics[width=165mm, scale=0.95]{lab04part11.png}
    \caption{Step response of system with time delay}
   \includegraphics[width=165mm,scale=0.85]{lab04part20.png}
   \caption{Simulink model of systems given in Part B}
  \includegraphics[width=165mm,scale=0.85]{lab04part21.png}
  \caption{Step response of system approximated by Half rule}
  \end{figure}
\section{Conclusion : }
\begin{itemize}
  \item In part A we get the more time delay in Half rule approximation, then in Taylor series
approximation in output response but settling time is almost same.
\item In part B we get little spike in actual response. But in approximation we do not get any
spike. Here FOPTD has more delay than SOPTD, and no changes in other characteristics.
\end{itemize}
\chapter{Experiment : 05}
\section{Title : Study of offset in a response of First order system with proportional controller, and
eliminating it using PI controller.}

\section{Apparatus :}
MATLAB/Simulink software

\section{Theory :}
there is a plant $G_p(s)$ has first order transfer function and controller
transfer function is
termed as $G_c(s)$, we are giving step response to the closed loop unity
feedback system in which,

\begin{equation}
  G_p(s) = \frac{K_p}{1 + s\tau_p},    G_c(s) = K_c(\frac{1+ s\tau_i}{s\tau_i})  
 \end{equation}\\
 \scalebox{1.25}{\bf Part A}
  \\
  first we apply proportional controller, i.e. $K_c$ to first order system and
  analyse offset which is difference between actual value and desired value of
  response.
  Now Closed loop transfer function in this case is,
 \begin{equation}
    H_{cl}(s)=\frac{C(s)}{R(s)} = \frac{K_c K_p}{1 + K_cK_p+ s T_p}
 \end{equation}
    here $K_p$ = 1, $T_p$ = 5,
    Applying Routh's stability criterion, we get stability condition as
\begin{align}
      T_p &> 0 \\
     (1 + K_pK_c) &> 0\\
     K_p K_c &> -1
\end{align}
here we need to change the value of $K_c$ = -2,-0.5,1,5,10 and analyse the
response of c(t), u(t),and its offset.

\noindent\scalebox{1.25}{\bf Part B}\par
\noindent In this part we are using PI controller to cotrol the behavious of plant tansfer
function and eliminate offset of proportional controller, so that closed
transfer function of entire system is

\begin{equation}
  Y_{cl}(s) = \frac{G_c(s)G_p(s)}{1 + {G_c(s)G_p(s)}}
\end{equation}\pagebreak
\\
  After substituting values
\begin{center}
  \scalebox{2.0}{$
    Y_{CL} = \frac{1 +s \tau_i}{\frac{\tau_p \tau_i}{K_c K_p}s^2 +\frac{\tau_i(1 + K_pK_c)}{K_pK_c}s + 1}
    $}
  \end{center}
  Now applying Routh Hurwitz's stability criteria to $Y_{CL}$ and $\tau_i \tau_p
  > 0$ so that
 \begin{align*}
    \frac{\tau_i \tau_p}{K_cK_p} &> 0 \\
    \frac{\tau_i(1+K_pK_c)}{K_cK_p} &> 0 \\
    K_p K_c &> 0    
 \end{align*}

here $\tau_p$ = 5, $K_i$ = 0.2, 4,($\tau_i$ = 0.25,5) and $K_c$ = 5 and obeserve that whether offset is
eliminated by PI only control or not.
\section{Observation :}
\begin{figure}[h!]
  \includegraphics[width = 165mm, scale = 0.85]{lab05part01.png}
  \caption{Simulink model of system for proportional control only}
\end{figure}
\begin{figure}[H]
  \includegraphics[width = 165mm, scale = 0.85]{lab05part11.png}
  \caption{step response of systems with different value of $K_c$}
  \includegraphics[width = 165mm, scale = 0.85]{lab05part12.png}
  \caption{controller response of systems with different value of $K_c$} 
\end{figure}
 \begin{figure}[H]
   \includegraphics[width = 165mm, height = 100mm, scale = 0.95]{lab05part20.png}
   \caption{simulink model of systems given in Part B}
   \includegraphics[width = 165mm,height = 65mm, scale = 0.85]{lab05part21.png}
   \caption{step response of systems with different value of $K_i$}
  \includegraphics[width = 165mm, height = 65mm, scale = 0.85]{lab05part22.png}
  \caption{controller response of systems with different value of $K_i$}
 \end{figure}







  
\section{Result :}


\begin{table}[H]
\centering
\resizebox{\textwidth}{!}{%
\begin{tabular}{@{}llll@{}}
\toprule
\rowcolor[HTML]{F0BB65} 
Value of Kc &
  \begin{tabular}[c]{@{}l@{}}Nature of response\\ C(t)\end{tabular} &
  \begin{tabular}[c]{@{}l@{}}Offset\\ E(s) = 1/(1+KpKc)\end{tabular} &
  \begin{tabular}[c]{@{}l@{}}Condition\\ of Kc \& Kp\end{tabular} \\ \midrule
10 & \begin{tabular}[c]{@{}l@{}}Stable and \\ Overdamped\end{tabular} & 0.091 & \begin{tabular}[c]{@{}l@{}}Same sign\\ Kp Kc \textgreater 0\end{tabular}   \\
5  & \begin{tabular}[c]{@{}l@{}}Stable and\\ Overdamped\end{tabular}  & 0.168 & \begin{tabular}[c]{@{}l@{}}Same sign\\ Kp Kc \textgreater 0\end{tabular}   \\
1  & \begin{tabular}[c]{@{}l@{}}Stable and\\ Overdamped\end{tabular}  & 0.5   & \begin{tabular}[c]{@{}l@{}}Same sign\\ Kp Kc \textgreater 0\end{tabular}   \\
-0.5 &
  \begin{tabular}[c]{@{}l@{}}system is conditiionally \\ stable\end{tabular} &
  2 &
  \begin{tabular}[c]{@{}l@{}}not same sign\\ -1\textless KpKc \textless 0\end{tabular} \\ \midrule
-2 & Unstable                                                         & --    & \begin{tabular}[c]{@{}l@{}}not same sign\\ Kp Kc \textless -1\end{tabular} \\ \bottomrule
\end{tabular}%
}
\caption{analysis of step response of system with different value of $K_C$}
\label{tab:my-table1}
\end{table}

In part A we analyzes various step response and get details about how closed
loop system responded at different value of $K_c$ and details of offset and
stability details given in table \par


\begin{table}[H]
\centering
\resizebox{\textwidth}{!}{%
\begin{tabular}{|l|l|l|l|}
\hline
\begin{tabular}[c]{@{}l@{}}Proportional Gain\\ Kc\end{tabular} &
  \begin{tabular}[c]{@{}l@{}}Integral gain \\ Ki\\ (1\textbackslash{}$\tau_i$)\end{tabular} &
  \begin{tabular}[c]{@{}l@{}}Output\\ Response C(t)\end{tabular} &
  \begin{tabular}[c]{@{}l@{}}speed of \\ controller\end{tabular} \\ \hline
5 & 0.2 & \begin{tabular}[c]{@{}l@{}}Stable and\\ No offset\end{tabular} & medium \\ \hline
5 & 4   & \begin{tabular}[c]{@{}l@{}}Stable and\\ No offset\end{tabular} & faster \\ \hline
1 & 0.2 & \begin{tabular}[c]{@{}l@{}}Stable and\\ Offset\end{tabular}    & slow   \\ \hline
\end{tabular}%
}
\caption{step response with varying $K_c$ and $K_i$ values}
\label{tab:my-table2}
\end{table}


In part B we are using PI only control to achieve errorless steady state
response
\pagebreak
\section{Conclusion : }
\begin{itemize}
\item by increasing value of proportional gain $K_c$ we get smaller offset but
    it has never becomes zero, it means some amount of band of error remains
    always in system in case of proportional only control.
\item Due to presense of integrator, past value of system responses takes
      into consideration and by using those value error becomes zero and also
      becomes negetive because of continuosly increasing of sum at zero error
      also.
\item to reduce steady state error we can use PI only control that will
        results satisfatorily, and achieves set point after 4 or 5 time contants
        but it will increase oscillation and system response becomes slower.
\end{itemize}

\chapter{Experiment : 06}
\section{Title : Stability analysis of a first order open loop unstable process with proportional control.}        
\section{Apparatus :}
MATLAB/Simulink Software

\section{Theory : }
take a unstable first order plant transfer function $G_p(s)$ and Controller $G_c(s)$
\begin{eqnarray}
  G_p(s) = \frac{1}{1-5s}\\
  \\
  G_c(s) = K_c
\end{eqnarray}
closed loop unity feedback transfer function is $Y_{CL}$, with step input
\begin{equation}
  Y_{CL} =\frac{C(s)}{R(s)} =  \frac{K_c}{(1 + K_c) -5s}
\end{equation}
Now applying RH's stability criteria to find condition for stability
\begin{align}
  if:  -5 &< 0 \\
  then:    Kc + 1 &< 0
\end{align}
then system is stable if $K_c$ < -1, Now observe C(t)=controlled output
and U(t)=controller output at value of different proportional gain of \[ K_c =
  1, -1, -10, -30\]
\section{Observation : }
\begin{figure}[t]
  \centering
   \includegraphics[width = 175mm, scale =0.95]{lab06part10.png}
   \caption{Simulink model of system for proportional control only}
\end{figure}
 \begin{figure}[H]
   \includegraphics[width = 165mm,  scale = 0.95]{lab06part11.png}
   \caption{step response of systems with different value of $K_c$}
 \end{figure}
 \begin{figure}[H]
   \includegraphics[width = 165mm, height= 90mm, scale = 0.95]{lab06part12.png}
   \caption{controller response of systems with different value of $K_c$}
 \end{figure}
\section{Result : }

\begin{table}[H]                                                                                                                                                       
\centering                                                                                                                                                             
\resizebox{\textwidth}{!}{%                                                                                                                                            
\begin{tabular}{@{}|l|l|l|l|@{}}                                                                                                                                       
\toprule                                                                                                                                                               
\begin{tabular}[c]{@{}l@{}}Proportional Gain\\ Kc\end{tabular} & Output Response & Offset & \begin{tabular}[c]{@{}l@{}}condition of\\ Kp \& Kc\end{tabular} \\ \midrule
1   & Unstable and Unbounded & infinite & Kp Kc \textgreater -1         \\ \midrule                                                                                    
-1  & Unstable and Unbounded & infinite & Kp Kc = -1                    \\ \midrule                                                                                    
-10 & Stable and Bounded     & -0.1     & Kp Kc \textless -1            \\ \midrule                                                                                    
-30 & Stable and Bounded     & -0.03    & Kp Kc \textless{}\textless -1 \\ \bottomrule                                                                                 
\end{tabular}%                                                                                                                                                         
}                                                                                                                                                                      
\caption{step response with varying $K_c$ value}                                                                                                                       
\label{tab:my-table}                                                                                                                                                   
\end{table}                                                                                                                                                            
                                                                                                                                                                       
\section{Conclusion : }
\begin{itemize}
  \item step response given to the unstable system gives unbounded output and
    approaches to infinite value
\item from RH's stability criteria we can say that to operate system in stable
  region we need to make $K_c$ < -1, or in more general way it is $K_c$ < 0
\item from varying value $K_c$ to negetive magnitude side we can observe that
  system stabilizes as closed loop right hand side pole effect is dilute huge and
  negetive gain value of $K_c$, in other sense we can get bounded output with
 some amount of offset.  
\end{itemize}

\chapter{Experiment : 07}
\section{Title : Application of DS method for PID controller settings for a
  second order plus time delay system}
\section{Aim : Design a PID controller for desired closed loop time constants}
\section{Apparatus :}
Matlab/Simulink Software
\section{Theory :}
In the Direct Synthesis (DS) method, the controller design is based on a process
model and a desired closed-loop transfer function.
The latter is usually specified for set-point changes, but responses to
disturbances can also be utilized.
Although these feedback controllers do not always have a PID structure, the DS
method does produce PI or PID controllers for common process models.
As a starting point for the analysis, consider the block diagramof a feedback
control system in Figure 7.1 The closed-loop transfer function for set-point
changes was derived
\begin{center}
  \includegraphics[scale = 0.8]{l7p1.png}
  \end{center}
$$\frac{Y}{Y_{sp}} = \frac{G_c G_p}{1 + G_c G_p}$$
Rearranging and sloving it we can get,
$$G_c = \frac{1}{G_p} \frac{Y/Y_{sp}}{1 - Y/Y_{sp}}$$
Also, it is useful to distinguish between the actual process $G_p$ and the
model,$\tilde{G_p}$, that provides an approximation of the process behavior.
A practical design equation can be derived by replacing the unknown $G_c$
by$\tilde{G_p}$, and Y/Ysp by a desired closed-loop transfer function,
$(Y/Y_{sp})_d$.
\begin{equation}
  G_c(s) = \frac{1}{\tilde{G_p}(s)} \frac{(Y/Y_{sp})_d}{1 - (Y/Y_{sp})_d}
\end{equation}
hence desired closed loop transfer function for process without time delays, the\\
{\bf first order model :}\\
\[(\frac{Y}{Y_{sp}})_d = \frac{1}{\tau_c s + 1}\]
  and now substituting above values in 7.1 and solving for $G_C(s)$,
\begin{equation}
  G_c(s) = \frac{1}{\tilde{G_p}}\frac{1}{\tau_c s}
\end{equation}
  the $\frac{1}{\tau_s} s$term provides integral control action and thus eliminate
  offset.\\
{\bf first order plus time delay model :}\\
 If the process transfer function contains a known time delay   ,a reasonable choice for the desired closed-loop transfer 
\[(\frac{Y}{Y_{sp}})_d = \frac{e^{- \theta s}}{\tau_c s + 1}\]
Combining above equation gives:
\begin{equation}
G_c(s) = \frac{1}{\tilde{G_p}\frac{e^{- \theta s}}{\tau_c s + 1 - e^{- \theta
        s}}
\end{equation}
now approximate time delay term as Taylor series expansion:
\[e^{- \theta s} \approx 1 - \theta s\]
and then re arranging above eq7.3 gives,
\begin{equation}
G_c(s)= \frac{(1 - \theta s)}{\tilde{G_p}s(\tau_c + \theta)}
\end{equation}
    Now consider standard FOPTD model as
    \[\tilde{G_p}(s) = \frac{K_p e^{- \theta s}}{\tau_p s + 1}\]
    Now standard PI controller equation as
    \[G_c = K_c (1 + \frac{1}{\tau_i s})\]
    now comparing it with eq 7.4 we can get values of \\
    $$k_c = \frac{\tau_p}{k_p(\theta + \tau_c)}$$\\
    and also $$ \tau_i = \tau_p $$
\\{\bf Second order plus time delay model :}\\
\begin{equation}
  \tilde{G_p}(s) = \frac{k_p e^{- \theta s}}{(\tau_1 s + 1)(\tau_2 s + 1)} 
\end{equation}
Now substituting into eq 7.4 and comparing with standard PID controller equation
as shown below
$$G_c = K_c(1  + \frac{1}{\tau_i s}+\tau_d s)$$
we can get value of
$$k_c = \frac{1}{k_p} \frac{\tau_1 + \tau_2}{\tau_c + \theta}$$ and also
$$\tau_i = \tau_1 + \tau_2$$
$$ \tau_d = \frac{\tau_1 \tau_2}{\tau_1 + \tau_2} $$
\section{Procedure :}
Now we have process transfer function for analysis as
$$ G_p(s) = \frac{2e^{-s}}{(10s + 1)(5s + 1)} $$
$$ \tilde{G_p}(s) = \frac{0.9 e^{-s}}{(10s + 1)(5s + 1)}$$

\begin{itemize}
\item Analyze step response when process is perfect$(G_p = \tilde{G_p})$
\item Analyse with model gain $k_p = 0.9$ instead of 2(not perfect)
\item when unit step change occured in both set point and disturbance, in
  disturbance it is occured at 80 sec
  \item simulate for 160 sec time and disturbance is introduced at 80 sec.
  \end{itemize}
  \pagebreak
\section{Observation :}
\begin{figure}[ht]
  \centering
  \includegraphics[width = 165mm, scale=0.8]{lab7p11.png}
  \caption{for K = 2, and $\tau_c$= 1,3, and 10}
  \includegraphics[width = 145mm, scale = 0.9]{lab7p12.png}
  \caption{step response with disturbance at 80 sec of second order system}
\end{figure}
\pagebreak
 \begin{figure}[H]
  \centering
  \includegraphics[width = 165mm, scale=0.8]{lab7p21.png}
  \caption{for K = 0.9, and $\tau_c$= 1,3, and 10}
  \includegraphics[width = 145mm, scale = 0.9]{lab7p22.png}
  \caption{step response with disturbance at 80 sec of second order system}
\end{figure}




\section{Results :}

% Please add the following required packages to your document preamble:
% \usepackage{graphicx}
\begin{table}[H]
\centering
\begin{tabular}{|l|l|l|l|}
\hline
                 & $\tau_c = 1$ & $\tau_c = 3$ & $\tau_c = 10$ \\ \hline
$k_c (k_p = 2)$  & 3.75         & 1.88         & 0.68          \\ \hline
$k_c(k_p = 0.9)$ & 8.33         & 4.17         & 1.51          \\ \hline
$\tau_i$         & 15           & 15           & 15            \\ \hline
$\tau_d$         & 3.33         & 3.33         & 3.33          \\ \hline
\end{tabular}%
\caption{Values of $k_c$ and $\tau_i$ and $\tau_d$}
\label{tab:exp7_results}
\end{table}
\section{Conclusion :}
\begin{itemize}
  \item value of $\tau_c$ decreases as $k_c$ increases or vice versa, but
    $\tau_i$ $\tau_d$ values fixed, deviation is large after disturbance
    occures.
  \item as $\tau_c$ increase response become sluggish
  \item in imperfact model oscillation starts quickly than perfact assumption
    \item decreasing $k_p$ values cause oscillatory response or make system
      unstable at instant hence less value than 0.9 make it to unstable
      operating point.
      
\chapter{Experiment : 08}
\section{Title : Internal Model Control (IMC) based PID controller design for
  $\Romannum{1}^{st}$ Order \& $\Romannum{2}^{nd}$ Order systems}

\section{Apparatus}
MATLAB/Simulink Software

\section{Theory}
\label{sec:lab8_theory}
The most common type of industrial controller is still the PID controller, and 
need of tuning and stabile performance is always desire. so for that we are
using here IMC based controller that can be formulated in the standard feedback
control structure and will result in equivalent PID controller.

\par

PID tuning parameters are tweaked on tranfer function model, but it is not
always clear how the process model effects the tuning decision. In the IMC
formulation the controller $G_c(s)$, is based directly on the ``good'' part of
the process transfer function. \par

we derive the feedback equivalence to IMC by using block diagram manipulation,
begin with IMC structure shown in fig8.1 \#A is converted to fig8.2 \#B and then
finally inner loop of the rearranged IMC structure shown in fig8.3 \#C, internal
loop with positive has transfer function like this. \\
let
\begin{align*}
  C(s) = Y(s) &= G_p(s) U(s)\\
  &= G_p(s) [G_c(s) e(s)]\\
  &= G_p(s) [G_c(s)(R(s)-Y(s) + \widetilde{Y}(s))]\\
  &= G_p(s)G_c(s)R(s) - G_p(s)G_c(s)Y(s) + G_p(s)G_c(s)U(s) \widetilde{G_p} (s)\\
  &= G_c(s)G_p(s)R(s) - Y(s)G_c(s)(G_p(s)-\widetilde{G_p}(s))\\
\\
  \frac{Y(s)}{R(s)} &= \frac{G_c(s)G_p(s)}{1 - G_c(s)(\widetilde{G_p}(s)-G_p(s))}
\end{align*}
Or also we can write for Internal loop,
\begin{equation}
  G_c^{\star} = \frac{U(s)}{R(s) - Y(s)} = \frac{G_c(s)}{1- \widetilde{G_p} (s) G_c(s) }
\end{equation}
where R(s)-Y(s) is error term used by a standard feedback controller.
\begin{figure}[H]
  \centering
  \includegraphics[width = 165mm, scale=0.9]{l8p11.png}
  \caption{Basic Block diagram of feedback conntrol system model using IMC}
  \includegraphics[width = 165mm, scale=0.9]{l8p1.png}
  \caption{}
  \includegraphics[width = 165mm, scale=0.9]{l8p2.png}
  \caption{Final FBC system model using IMC}
\end{figure}

\subsection{Part 1: IMC Based PID Design for a First Order Process}
Find the PID equivalent to IMC for a first-order process
\begin{equation}
  \widetilde{G_p} (s) = \frac{k_p}{\tau_p s + 1}
\end{equation}
{\bf Step 1:}
Develop a process model $\widetilde{G_p}$(s) and factor it into invertible(Good) and
Non-invertible (Bad) portions.
\[\widetilde{G_p}(s) = \widetilde{G_{p-}}(s) \widetilde{G_{p+}(s)} \]
{\bf Step 2:}
Ideal IMC controller will be inverse of invertible part of {$\widetilde{G_c}$(s)}
\[ \therefore \widetilde{G_c}(s) = (\widetilde{G_p}(s))^{-1}   \]
{\bf Step 3:}
Find the IMC controller transfer function, $G_c(s)$, which includes
a filter to make $G_c(s)$ semi proper \\
\begin{equation}
{\bf  G_c(s) = \widetilde {G_c}(s) f(s) =  G_{p-}^{-1}(s)f(s) = \frac{\tau _ps +1}{k_p} \frac{1}{\lambda s + 1}}
\end{equation}
so that \[G_c(s)  = \frac{\tau_p s + 1}{k_p(\lambda s + 1)}\]
{\bf Step 4:}
Find the equivalent standard feedback controller using the transformation \\
\begin{equation*}
  G_c^{\star}(s) = \frac{G_c(s)}{1 - \widetilde{G_p}(s)G_c(s)}
\end{equation*}
\begin{center}
\scalebox{2.0}{$
= \frac { \frac{\tau_ps + 1}{k_p(\lambda s + 1)}}{1 - \frac{k_p(\tau_ps +
    1)}{k_p(\tau_p s + 1)(\lambda s + 1)}}$}
\end{center}
\begin{equation}
       G_c^{\star}(s)= \frac{\tau_ps + 1}{k_p \lambda s}
\end{equation}
{\bf Step 6:}
Rearrange above equation as per time constant form of PI controller as shown
below
\begin{align*}
 &= k_c ( 1 + \frac{1}{\tau_i s}) \\
   G_c^{\star}(s)  &= \frac{\tau_p}{k_p \lambda} + \frac{1}{sk_p
                    \lambda} \\
  k_c &= \frac{\tau_p}{k_p \lambda} \\
  \tau_i &= \tau_p
\end{align*}

\noindent For analysis of that we have taken plant transfer function\\
\[G_p(s) = \frac{2}{1+10s}\]\\
so here we have value of $k_p = 2, \tau_p = 10$ and $\lambda = 2 \& 5$, and then we get
the value of $k_c$ = 2.5 \& 1, and $\tau_i = 10$.
we have analyse step response of IMC based PI controlled first order system
with different value of \lambda.
\subsection{Part 2: IMC based PID design for Second Order Process}
 Find the PID equivalent to IMC for a Second-order process
 \begin{equation}
   \widetilde{G_p} (s) = \frac{k_p}{(\tau_1 s + 1)(\tau_2 s + 1)}
 \end{equation}
 {\bf Step 1:}
 Develop a process model $\widetilde{G_p}$(s) and factor it into invertible(Good) and
 Non-invertible (Bad) portions.
 \[\widetilde{G_p}(s) = \widetilde{G_{p-}}(s) \widetilde{G_{p+}(s)} \]
 {\bf Step 2:}
 Ideal IMC controller will be inverse of invertible part of {$\widetilde{G_c}$(s)}
 \[ \widetilde{G_c}(s) = (\widetilde{G_p}(s))^{-1}   \]
 {\bf Step 3:}
 Find the IMC controller transfer function, $G_c(s)$, which includes
 a filter to make $G_c(s)$ semi proper \\
 \begin{equation}
 {\bf  G_c(s) = \widetilde {G_c}(s) f(s) =  G_{p-}^{-1}(s)f(s) = \frac{(\tau _1s +1)(\tau_2s+1)}{k_p}\frac{1}{1 + s\lambda}}
 \end{equation}
 {\bf Step 4:}
 Find the equivalent standard feedback controller using the transformation \\
 \begin{equation*}
   G_c^{\star}(s) = \frac{G_c(s)}{1 - \widetilde{G_p}(s)G_c(s)}
 \end{equation*}
 \begin{center}
   \scalebox{1.5}{$
     G_c^{\star}(s) = \frac{\tau_1 \tau_2 s^2 + (\tau_1 + \tau_2)s
       +1}{k_p\lambda s}
     $}
 \end{center}
 \begin{equation}
        G_c^{\star}(s)= \frac{(\tau_1 + \tau_2)}{k_p \lambda} + \frac{1}{k_p \lambda s} + \frac{\tau_1 \tau_2 s }{k_p \lambda}
 \end{equation}
 {\bf Step 6:}
 Rearrange above equation as per time constant form of PID controller as shown
 below
 \begin{align*}
    &= k_c ( 1 + \frac{1}{\tau_i s} + s \tau_d) \\
   k_c &= \frac{\tau_1 + \tau_2}{k_p \lambda} \\
   \tau_i &= \tau_1 + \tau_2 \\
   \tau_d &= \frac{\tau_1 \tau_2}{\tau_1 + \tau_2}
 \end{align*}

 \noindent For analysis of that we have taken plant transfer function\\
 \[G_p(s) = \frac{2}{(1+10s)(1+5s)}\]\\
 so here we have value of $k_p = 2, \tau_1 = 10, \tau_2 = 5$ and $\lambda = 2 \& 5$, and then we get
 the value of $k_c$ = 3.75 \& 1.5, and $\tau_i = 15, \& \tau_d = 3.33$.
 we have analyse step response of IMC based PID controlled second order system
 with different value of $\lambda$.
 \pagebreak
 \section{Observation :}
{\bf Part 1}
\begin{figure}[H]
  \centering
   \includegraphics[width = 130mm, scale = 0.9]{lab8p11.png}
   \includegraphics[width = 130mm, scale = 0.9]{lab8p12.png}
   \caption{Step response of IMC based PI control of First order system}
 \end{figure}
 \\
 \pagebreak

 {\bf Part 2}
 
 \begin{figure}[H]
   \centering
   \includegraphics[width = 130mm, scale =0.9]{lab8p21.png}
   \includegraphics[width = 130mm, scale = 0.9]{lab8p22.png}
   \caption{Step response of IMC based PID control of Second Order System}
   \end{figure}
   \section{Results :}

\begin{table}[H]
\centering
\begin{tabular}{|l|l|l|}
\hline
                 & $k_c$ & $\tau_i$\\ \hline 
$\lambda = 2$  & 2.5         & 10   \\ \hline
$\lambda = 5$  & 1         & 10    \\ \hline
\end{tabular}%
\caption{Values of $k_c$ and $\tau_i$}
\label{tab:exp8a_results}
\end{table}

\begin{table}[H]
\centering
\begin{tabular}{|l|l|l|l|}
\hline
                 & $k_c$ & $\tau_i$ & $\tau_d$ \\ \hline
$\lambda = 2$  & 3.75         & 15         & 3.33          \\ \hline
$\lambda = 5$ & 1.5         & 15         & 3.33        \\ \hline
\end{tabular}%
\caption{Values of $k_c$ and $\tau_i$ and $\tau_d$}
\label{tab:exp8b_results}
\end{table}
\pagebreak
\section{Conclusion :}
\begin{itemize}
  \item in part 1 : we get offset in output without controller so using IMC
    based controller we get better response (IMC based using PI only)
  \item as value of $\lambda$ is increasing settling time increasing
    \item in part 2 of second order system PID controller removes offset
      efficiently and output response is nicely tracking set point
      \item on changing value of $\lambda$ only $k_c$ changes other have no
        effect and settling time also increases in case of second order system
        also as $\lambda$ increases
        \end{itemize}
 \chapter{Experiment : 09}
 \section{IMC based PID Controller design for Systems with dead time, and
   Unstable system}
 \section{Apparatus :}
 MATLAB/Simulink
 \section{Theory :}
 In order to arrive at a PID equivalent form for processes with a time-delay, we must make some
approximation to the deadtime,so We will use a first-order Pad� approximation
for deadtime.
$\Romannum{1}^{st}$ order pad\'e approx as
\[e^{-\theta s} = \frac{1- 0.5 \theta s}{1 + 0.5 \theta s}\]
\subsection{Part 1 : FOPTD system}

Now lets take, First Order Plus Delay Time FOPTD system:
\[ \widetilde{G_p}(s) = \frac{k_p e^{- \theta s}}{1 + s \tau_p}\]

\noindent{\bf Step : 1}
\[\widetilde{G_p}(s) = \frac{k_p(1 - 0.5 \theta s)}{(1 + s \tau_p)(1 + 0.5
    \theta s)}\]
{\bf Step : 2}
Factor out non invertible terms(stable part or LHS part)
\[\widetilde{G_{p-}}(s) = \frac{k_p}{(1 + s \tau_p)(1+ 0.5 \theta s)}\]
{\bf Step : 3}
Ideal IMC controller as

\[\widetilde{G_c}(s) = \frac{(1 + s \tau_p)(1 + 0.05 \theta s)}{k_p}\]
{\bf Step : 4}
IMC controller with filter
\[G_c(s) = \frac{(1 + s \tau_p)(1 + 0.05 \theta s)}{k_p} \frac{1}{1 + \lambda
    s}\]
{\bf Step : 5}
\[G_c^{\star}(s) = \frac{G_c(s)}{1 - \widetilde{G_p}(s) G_c(s)}\]
\\
\begin{center}
\scalebox{1.70}{\[G_c^{\star} = \frac{\frac{(1 + s \tau_p)(1 + 0.5 \theta s)}{k_p (1 + \lambda
      s)}}{1 - \frac{(k_p(1-0.5 \theta s))(1 + s \tau_p)(1 + 0.5 \theta s)}{(1 +
      s\tau_p)(1 + s \theta 0.5)k_p(1 + \lambda s)}}\]}
\end{center}
\[ = \frac{(1 + s \tau_p)(1 + 0.5 \theta s)}{k_p(1 + \lambda s) - k_p(1 - 0.5
    \theta s)} \]
\[ = \frac{s^2 \tau_p 0.5 \theta s + s(\tau_p + 0.5 \theta) + 1}{sk_p(\lambda + 0.5
    \theta)} \]
\[ = \frac{(\tau_p + 0.5 \theta)}{k_p(\lambda + 0.5 \theta)} +
  \frac{1}{s(k_p(\lambda + 0.5 \theta))}+ \frac{s(\tau_p 0.5
    \theta)}{k_p(\lambda + 0.5 \theta)}\]
\\
{\bf Step : 6}
Comparing it with Ideal PID controller equation we can get values of
$k_c$,$\tau_i$, $\tau_d$,
\[
  k_c = \frac{\tau_p + 0.5 \theta}{k_p(\lambda + 0.5 \theta)}\]
\[ \tau_i = \tau_p
+0.5 \theta)\]
\[ \tau_d = \frac{\theta \tau_p}{2 \tau_p + \theta} \]
now we have plant transfer function with time dalay as
\[\widetilde{G_p}(s) = \frac{2 e^{-5s}}{1 + 10s}\]
so from that we can get $k_p = 2$, $\theta = 5$, and $\tau_p = 10$, 
Now substituting above values in eq() we can get $k_c = 0.833$, $\tau_i = 12.5$,
and $\tau_d = 2$.
here we have assume value of $\lambda = \theta$ for study of performance but
there is value of  $ \lambda > 0.8 \theta $ is conclude by other researchers
experiment that will results in more accurate response of systems.  
\subsection{Part 2 : Unstable System}
For unstable process more complicated filter transfer function required, so here
we have first order unstable process is
\[ \widetilde{G_p}(s) = \frac{k_p}{1 -\tau_u s}\]
{\bf Step 1 :}
Find the IMC controller transfer function, $G_c(s)$

here we also need to make controller transfer function to semi proper so that,
$f(s) \mid _{s = p_u}$= 1
\\
\begin{equation*}
  f(s) = \frac{1 + s \gamma}{(1 + s \lambda)^n}
\end{equation*}
value of $\gamma$ that satisfies $$f(s=p_u) =1$$
$$ P_u = \frac{1}{\tau_u}$$
$$G_c(s) = \frac{(1 - \tau_u s) (\gamma s + 1)}{k_p(\lambda s +1)^2}\mid_{n=2}$$
\\
$$f(s) = \frac{\gamma s + 1}{k_p (\lambda s + 1)^2} = 1 }$$

after solving, we can get
$$ \gamma = \lambda [\frac{\lambda}{\tau_u} + 2]$$
{\bf Step 2:}
Find the equivalent standard feedback controller using the transformation\\

$$G_c^{\star}   = \frac{G_c(s)}{1 - \widetilde{G_p}(s) G_c(s)}$$
\begin{center}
\scalebox{1.55}{
$ = \frac{\frac{1 - \tau_u s}{k_p} \frac{\gamma s + 1}{(\lambda s +1)^2}}{1 -
   \frac{k_p}{1 - \tau_u s}\frac{1 - \tau_u s}{k_p} \frac{\gamma s + 1}{(\lambda
     s +1)^2}} $}
\end{center}
\\
$$ = \frac{(1 - \tau_u s)(\gamma s + 1)}{k_p(\lambda^2 s^2 + 2 \lambda s - \gamma
  s)}$$
\\
multiply and divide with $\gamma s$
\begin{center}


\scalebox{1.45}{
$ = \frac{\frac{1}{k_p}(1 - \tau_u s + 1)(1 + \gamma s)}{\lambda^2 s^2 + s(2
  \lambda - \gamma )} \frac{\gamma s}{\gamma s}$}\\
\\
\end{center}
by simplifying it
\begin{center}\\
\scalebox{1.45}{
$ = \frac{\frac{1}{k_p} (1 - \tau_u s) \frac{\gamma s + 1}{\gamma
    s}}{\frac{\lambda^2 s}{\gamma} + \frac{(2 \lambda - \gamma)}{\gamma}}$}\\
\end{center}
and then we can get,
\begin{center}
\scalebox{1.6}{$ = \frac{\frac{\gamma (1 - \tau_u s)(\gamma s + 1)}{k_p \gamma s(2 \lambda -
      \gamma)}}{\frac{\lambda^2 s}{2 \lambda - \gamma} + 1}$}\\
\end{center}
Now substituting value of $\gamma$ as
\begin{center}
$$\gamma &= \lambda (\frac{\lambda }{\tau_u} + 2 )$$\\
\scalebox{1.55}{
$G_c^{\star} = \frac{\frac{\gamma(1 - \tau_u s)(1  + \gamma s)}{k_p \gamma s(2 \lambda - \gamma)}}{\frac{\lambda^2}{2 \lambda - (\frac{\lambda^2}{\tau_u} + 2 \lambda)}s + 1}$}
\end{center}\\
$$ G_c^{\star} = \frac{\gamma}{k_p(2 \lambda - \gamma)}\frac{\gamma s + 1}{\gamma
  s}$$\\
{\bf Step : 3}
This is in form of PI controller, in which
\begin{eqnarray}
  k_c = \frac{\gamma}{k_p(2 \lambda - \gamma)}\\
\tau_i = \gamma
\end{eqnarray}
here we have $k_p = 1$, $\tau_p =1$, and we have to analyse all responses with
different $\gamma$ values, like $\gamma = 0.5, 1, 2$, by substituting above
values in above equation we can get values
for $\lambda = 0.5$ as $k_c = -1$ , $\gamma = \tau_i = 1.25$
for $\lambda = 1$ we get $k_c = -1$, $\tau_i = 3$
for $\lambda = 2$ we get $k_c = -2$, $\tau_i = 8$
\section{Observation :}
 {\bf Part 1}
 \begin{figure}[H]
   \centering
   \includegraphics[width = 130mm, scale = 0.9]{lab9p11.png}
   \\
    \includegraphics[width = 130mm, scale = 0.9]{lab9p12.png}
    \caption{Step response of IMC based PID control of First order plus time
      delay system}
  \end{figure}

  \\
  \\
  {\bf Part 2}

  \begin{figure}[H]
    \centering
    \includegraphics[width = 130mm, scale =0.9]{lab9p21.png}
    \includegraphics[width = 130mm, scale = 0.9]{lab9p22.png}
    \caption{Step response of IMC based PI control of Unstalbe System}
    \end{figure}
    \section{Results :}

 \begin{table}[H]
 \centering
 \begin{tabular}{|l|l|l|l|}
 \hline
                  & $k_c$ & $\tau_i$ & $\tau_d$ \\ \hline
 $\lambda = 5$  & 0.833         & 12.5         & 2          \\ \hline
 $\lambda = 10$  & 0.5         & 12.5         & 2          \\ \hline  
 $\lambda = 15$ & 0.35        & 12.5         & 2        \\ \hline
 \end{tabular}%
 \caption{Values of $k_c$ and $\tau_i$ and $\tau_d$}                    
 \label{tab:exp9a_results}
\end{table}

 \begin{table}[H]
 \centering
 \begin{tabular}{|l|l|l|}
 \hline
                  & $k_c$      & $\tau_i$\\ \hline
 $\lambda = 0.5$  & -5         & 1.25     \\ \hline
   $\lambda = 1$    & -3         & 3        \\ \hline
 $\lambda = 2$    & -2         & 8        \\ \hline
                                                               
 \end{tabular}%
 \caption{Values of $k_c$ and $\tau_i$}
 \label{tab:exp9b_results}
 \end{table}



 \\
 \\
 \\

 \section{Conclusion :}
 \begin{itemize}
      \item in part 1 : oscillatory response and offset both can be eliminated
        at desired level using IMC based PID controller in FOPTD as $\lambda$
        values are increased response smooth out or good set point tracking
        achieved
        \item in part 2 : in unstable process we need to choose the desired
          value for $\lambda$ as a trade-off between performance and robustness and it
          is achieved better response as $\lambda$ uncreases peak value increases. 
\end{itemize}
\chapter{Experiment : 10}
\section{Design of PID Controller based on Zeigler Nichols closed loop
  oscillation \& Tyreus Luyben method}
\section{Aim:}
\begin{itemize}
\item Design PID controller based on Ziegler Nichols \& Tyreus Luyben method
\item Evaluate performance criteria using IAE, ISE, \& ITAE
\end{itemize}
\section{Apparatus}
Matlab  Simulink Software
\section{Theory:}
Let's start with closed loop system as shown in figure
\begin{center}
\includegraphics[width = 120mm, scale = 0.9]{lab10f1.png}
\end{center}
\par
So here we have plant with proportional controller {\bf K} in unity feedback
closed loop configuration in which we have to increase value of K upto $K_u$
where ,$K_u$ is value at firstly sustained oscillation occures.
\par
Now find value of $\tau_u$ from steady state sustained oscillation condition, where
$\tau_u$ is time period in seconds for oscillating waveform
\par
For slightly larger
values of controller gain, the closed-loop system is unstable, while for slightly
lower values the system is stable. so we can choose the values of controller as
per table shown below for which we first need to calculate $K_u$ and $\tau_u$
for particular system by using routh-array stability criteria or by any means. 

{\bf based on Ziegler Nichols method}
\begin{table}[H]
 \centering
\begin{tabular}{@{}llll@{}}
\toprule
\begin{tabular}[c]{@{}l@{}}Type of \\ Controller\end{tabular} & $K_c$                          & $\tau_i$                                   & $\tau_d$\\ \midrule
\multicolumn{1}{|l|}{P}                                       & \multicolumn{1}{l|}{$0.5K_u$}  & \multicolumn{1}{l|}{$\infty$}              & \multicolumn{1}{l|}{0} \\ \midrule
\multicolumn{1}{|l|}{PI}                                      & \multicolumn{1}{l|}{$0.45K_u$} & \multicolumn{1}{l|}{$\frac{1}{1.2\tau_u}$} & \multicolumn{1}{l|}{0} \\ \midrule
\multicolumn{1}{|l|}{PID}                                     & \multicolumn{1}{l|}{$0.6K_u$}  & \multicolumn{1}{l|}{$\frac{\tau_u}{2}$}     & \multicolumn{1}{l|}{$\frac{\tau_u}{8}$}     \\ \midrule
\end{tabular}
\end{table}
\par
Tyreus and Luyben have suggested tuning parameter rules that result in less
oscillatory responses and that are less sensitive to changes in the process
condition. so we can tune PI , and PID controller by putting values of $K_u$ and
$\tau_u$ in below table.



{\bf Based on Tyreus Luyben method}
\begin{table}[H]
\centering
\begin{tabular}{@{}llll@{}}
\toprule
\begin{tabular}[c]{@{}l@{}}Type of \\ Controller\end{tabular} & $K_c$                          & $\tau_i$                                   & $\tau_d$\\ \midrule
 \multicolumn{1}{|l|}{P}                                       & \multicolumn{1}{l|}{$0.5K_u$}  & \multicolumn{1}{l|}{$\infty$}              & \multicolumn{1}{l|}{0} \\ \midrule
 \multicolumn{1}{|l|}{PI}                                      & \multicolumn{1}{l|}{$\frac{K_u}{3.2}$} & \multicolumn{1}{l|}{$2.2 \tau_u$} & \multicolumn{1}{l|}{0} \\ \midrule
 \multicolumn{1}{|l|}{PID}                                     & \multicolumn{1}{l|}{$\frac{K_u}{2.2}$}  & \multicolumn{1}{l|}{$2.2 \tau_u$}     & \multicolumn{1}{l|}{$\frac{\tau_u}{6.3}$}     \\ \midrule
\end{tabular}
\end{table}
\\
\\
\\
\par
So to compare effectiveness and evaluate its performance we need to use some
criteria IAE, ISE, \& ITAE as shown below
\\
\begin{itemize}
\item IAE = Integral Absolute Error $\int_{0}^{\infty}|e(t)|\,dt$
\item ISE = Integral Square Error $\int_{0}^{\infty}e^{2}(t)\,dt$
  \item ITAE = Integral Time Absolute Error $\int_{0}^{\infty}t|e(t)|\,dt$
  \end{itemize}

\section{Procedure}
we have plant transfer function as
\[ G(s) = \frac{1}{6s^3 + 11s^2 + 6s + 1}\]
and by using routh array stability criteria we can get value of $K_u = 10$ and
$\tau_u= 6.28$
and then evaluate above system using Ziegler Nichols and Tyreus Luyben method
and find the value of IAE, ISE, and ITAE for each controller tuning methods.

\section{Observation}
\begin{center}
  \includegraphics[width = 165mm, scale = 0.9]{lab10fig1.png}
  Fig 1: System under Observation
  \includegraphics[width = 165mm, scale = 0.9]{lab10fig2.png}
  Fig 2: Responses  with the ZN method
  \includegraphics[width = 165mm, scale = 0.9]{lab10fig3.png}
  Fig 3: Responses with TL method
\section{Results}
Evalution of IAE, ISE, \& ITAE of system tuned by ZN method
\begin{table}[H]
\centering
\begin{tabular}{@{}llll@{}}
\toprule
\begin{tabular}[c]{@{}l@{}}Type of \\ Controller\end{tabular} & IAE                          & ISE                                   & ITAE \\ \midrule
 \multicolumn{1}{|l|}{P}                                       & \multicolumn{1}{l|}{10.35}  & \multicolumn{1}{l|}{3.415}              & \multicolumn{1}{l|}{213.4} \\ \midrule
 \multicolumn{1}{|l|}{PI}                                      & \multicolumn{1}{l|}{9.395} & \multicolumn{1}{l|}{3.886} & \multicolumn{1}{l|}{126.4} \\ \midrule
 \multicolumn{1}{|l|}{PID}                                     & \multicolumn{1}{l|}{3.125}  & \multicolumn{1}{l|}{1.443}     & \multicolumn{1}{l|}{13.12}     \\ \midrule
\end{tabular}
\end{table}
\pagebreak
Evalution of IAE, ISE, \& ITAE of system tuned by TL method
\begin{table}[H]
\centering
\begin{tabular}{@{}llll@{}}
\toprule
\begin{tabular}[c]{@{}l@{}}Type of \\ Controller\end{tabular} & IAE                          & ISE                                   & ITAE \\ \midrule
 \multicolumn{1}{|l|}{PI}                                      & \multicolumn{1}{l|}{8.593} & \multicolumn{1}{l|}{3.074} & \multicolumn{1}{l|}{161.6} \\ \midrule
 \multicolumn{1}{|l|}{PID}                                     & \multicolumn{1}{l|}{3.145}  & \multicolumn{1}{l|}{1.207}     & \multicolumn{1}{l|}{28.7}     \\ \midrule
\end{tabular}
\end{table}

\section{Conclusion}
\begin{itemize}
\item closed-loop behavior tends to be oscillatory and sensitive to
    uncertainty with P and PI only control because for finding tuning parameter
    we need to drive system at point where system have sustained oscillation
\item The tuning parameters are also not very robust, that is, they are very
sensitive to process uncertainty. If the process conditions change, then the control
system may become unstable.
\item Tyreus-Luyben parameters result in less oscillatory responses and will be less
sensitive to uncertainty in compare to ZN tuning method. 
  \end{itemize}
\end{document}
